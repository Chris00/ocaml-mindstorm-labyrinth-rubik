%\documentclass[handout,compress]{beamer}
\documentclass[compress]{beamer}

\usepackage[latin1]{inputenc}
\usepackage[francais]{babel}
\usepackage{textcomp}
\usepackage{times,mathptmx,bbm}
\usepackage{graphicx,tikz}
\usepackage{array}

%%%%%%%%%%%%%%%%%%%%%%%%%%%%%%%%%%%%%%%%%%%%%%%%%%%%%%%%%%%%%%%%%%%%%%
\makeatletter
\errorcontextlines=5

%\setbeameroption{show notes on second screen}

\setlength{\overfullrule}{10pt}

\AtBeginDocument{
  \setlength{\abovedisplayskip}{3pt plus 1pt}
  \setlength{\belowdisplayskip}{\abovedisplayskip}
}

\definecolor{mygreen}{RGB}{23, 120, 49}
\definecolor{lightgreen}{RGB}{238, 255, 151}
%\definecolor{umhblue}{RGB}{1, 56, 147}
\definecolor{umhblue}{RGB}{51, 106, 207}
%\definecolor{thm title}{RGB}{172, 10, 45}
\definecolor{thm title}{RGB}{124, 4, 30}
\definecolor{orangered}{RGB}{245, 94, 39}
\definecolor{water}{RGB}{200,193,251}

\definecolor{darkgreen}{RGB}{0, 136, 7}
\definecolor{em}{RGB}{245, 94, 39}
\definecolor{level}{RGB}{204,255,224}
\definecolor{problem}{rgb}{0.84,0.5,0}

\mode<presentation|handout>{%
  \usetheme{Boadilla}
%  \useoutertheme[subsection=false]{miniframes}
  \useinnertheme[shadow]{rounded}
  \setbeamercovered{transparent=1}
  \usecolortheme[named=umhblue]{structure}
  \setbeamercolor{block title}{bg=thm title,fg=yellow!15}
  \setbeamercolor{block body}{bg=yellow!15}
  \setbeamercolor{block title example}{bg=umhblue}
  \setbeamercolor{block body example}{bg=yellow!15}
  \setbeamercolor{alerted text}{fg=orangered}
}

\mode<presentation|handout:0> {%
  \AtBeginDocument{%
    \pgfdeclareverticalshading{beamer@headfade}{\paperwidth}
    {%
      color(0cm)=(bg);
      color(1cm)=(umhblue!20)%
     }
  }
  \addtoheadtemplate{\pgfuseshading{beamer@headfade}\vskip-1cm}{}
}

\mode<handout>
{
  \pgfpagesuselayout{4 on 1}[a4paper, border shrink=5mm,landscape]
}


%% Math

\@ifundefined{leqslant}{}{\let\le=\leqslant}
\@ifundefined{geqslant}{}{\let\ge=\geqslant}
\@ifundefined{subseteq}{}{\let\subset=\subseteq}

\newcommand{\IR}{{\mathbb R}}
%\newcommand{\id}{{\mathbbm 1}}
\newcommand{\id}{{\mathbf 1}}
\newcommand{\G}{{\mathcal G}}

\newcommand{\cube}[1]{%
  \begin{tikzpicture}[x=0.7ex,y=0.7ex, baseline=2ex]
    \input{#1}
  \end{tikzpicture}
}


\makeatother
%%%%%%%%%%%%%%%%%%%%%%%%%%%%%%%%%%%%%%%%%%%%%%%%%%%%%%%%%%%%%%%%%%%%%%

% \pgfdeclareimage[width=6mm]{UMH-logo}{Acad-logoumh}
% \pgfdeclareimage[width=6mm]{Acad-logo}{Acad-logo+txt}

\title{Atelier de robotique}
\subtitle{%Comment sortir d'un labyrinthe ?\\
  Comment résoudre le Rubik's cube ?}
\author[Institut de Mathématique]{%
  Marc \textsc{Ducobu}
  \and Dany \textsc{Maslowski}
  \and Julie \textsc{De Pril}
  \and Christophe~\textsc{Troestler}}
%\date{PDS, mars 2008}
\date{PDS, \today}
\institute[UMONS]{Université de Mons\\[3pt]
  %\pgfuseimage{UMH-logo}\qquad\pgfuseimage{Acad-logo}%
  %\includegraphics[width=10mm]{Acad-logoumh}\qquad
  %\includegraphics[width=10mm]{Acad-logo+txt}
  \includegraphics[width=20mm]{UMONS+txt}
}
%\logo{\pgfuseimage{UMH-logo}}



\begin{document}

\begin{frame}[plain]
  \titlepage
\end{frame}

\section*{Plan}
\begin{frame} \frametitle{Plan}
  \tableofcontents[subsectionstyle=hide]
\end{frame}

% Besoin d'un slide préliminaire pour expliquer ce qu'est un Rubik's
% cube et comment ça marche?


\section{Préliminaires}
\begin{frame} \frametitle{Appellation des faces du Rubik's cube}
  % dessin d'un Rubik's cube avec le nom des faces (F,B,U,D,L,R)
  \begin{tikzpicture}[x=3ex,y=3ex]
    \input{expose-id.tex}
    \only<2->{%
      % Cover with a transparent white to have smooter colors
      \fill[semitransparent,white] (0,0) rectangle (11.14,8.14);
      % Label faces
      \node at (4.5,4.5) {$F$};
      \node at (9.62, 6.62) {$B$};
      \node at (4.5, 1.5) {$D$};
      \node at (1.5, 4.5) {$L$};
      \node[rotate=-13] at (5.5, 7.02) {$U$};
      \node[rotate=10] at (7, 5.5) {$R$};
    }
  \end{tikzpicture}
  \hfil
  \begin{minipage}[b]{0.4\linewidth}
    Rubik's cube en «~position de base~», appelé le Rubik's cube
    \textit{identité}, noté $\id$.

    \onslide<2->{%
    \begin{align*}
      F &= \text{Front}\\
      B &= \text{Back}\\
      L &= \text{Left}\\
      R &= \text{Right}\\
      U &= \text{Up}\\
      D &= \text{Down}
    \end{align*}
  }
  \end{minipage}
\end{frame}

\section{Mouvements}
\begin{frame} \frametitle{Mouvements}

  \noindent
  \begin{minipage}[b]{0.56\linewidth}
    \begin{tikzpicture}[x=3ex,y=3ex]
      \only<1>{%
        \input{expose-id.tex}
        \node at (4.5,4.5) {$F$};
        \draw[dashed] (3,6) -- (4.5,4.5) -- (6,6);
        \draw[->,thick] (3.8, 5.2) arc(135:45: 1);
      }
      \only<2>{
        \input{expose-F1.tex}
        \node[rotate=-90] at (4.5,4.5) {$F$};}
      \only<3>{
        \input{expose-F2.tex}
        \node[rotate=-180] at (4.5,4.5) {$F$};}
      \only<4>{
        \input{expose-F3.tex}
        \node[rotate=90] at (4.5,4.5) {$F$};}
      \only<5>{
        \input{expose-id.tex}
        \node at (4.5,4.5) {$F$};}
    \end{tikzpicture}
    \par\vspace{1ex}

    $F ={}$faire tourner la face $F$ de 90°.

    \begin{block}{}
      Tourner une face d'un multiple de 90° degrés est appelé un
      \textit{mouvement}.
    \end{block}
  \end{minipage}
  \hspace*{5ex}%
  \begin{minipage}[b]{0.3\linewidth}

    \begin{tikzpicture}[x=0.8ex,y=0.8ex]
      \input{expose-id.tex}
      \node[right] at (13, 4.5) {$\id$};
      \onslide<2->%
      \begin{scope}[yshift=-8ex]
        \input{expose-F1.tex}
        \node[right] at (13, 4.5) {$F$};
      \end{scope}
      \onslide<3->%
      \begin{scope}[yshift=-16ex]
        \input{expose-F2.tex}
        \node[right] at (13, 4.5) {$F^2$};
      \end{scope}
      \onslide<4->%
      \begin{scope}[yshift=-24ex]
        \input{expose-F3.tex}
        \node[right] at (13, 4.5) {$F^3$};
      \end{scope}
      \onslide<5->%
      \begin{scope}[yshift=-32ex]
        \input{expose-id.tex}
        \node[right] at (13, 4.5) {$F^4 = \id$};
      \end{scope}
    \end{tikzpicture}

  \end{minipage}
  \note{ Quand on tourne une face de 360°, c'est comme si l'on avait
    rien fait. En effet, nous retrouvons le Rubik's cube de départ.  }
\end{frame}

\begin{frame} \frametitle{Mouvements de base}

  \begin{block}{18 mouvements de base}
    $F, F^2, F^3, B, B^2, B^3, L, L^2, L^3, R, R^2, R^3, U, U^2, U^3,
    D, D^2, D^3$
  \end{block}
  où $X^i$ signifie \og tourner la face $X$ de $i$ fois 90°
  dans le sens horloger\fg.

  \begin{center}
    % Same figure of slide 1
    \begin{tikzpicture}[x=3ex,y=3ex]
      \input{expose-id.tex}
      \fill[semitransparent,white] (0,0) rectangle (11.14,8.14);
      % Label faces
      \node at (4.5,4.5) {$F$};
      \node at (9.62, 6.62) {$B$};
      \node at (4.5, 1.5) {$D$};
      \node at (1.5, 4.5) {$L$};
      \node[rotate=-13] at (5.5, 7.02) {$U$};
      \node[rotate=10] at (7, 5.5) {$R$};
  \end{tikzpicture}
  \end{center}
\end{frame}

\begin{frame} \frametitle{\og Multiplication\fg{} de mouvements}
  \begin{tikzpicture}[x=2.6ex,y=2.6ex]
    \only<1>{\input{expose-id.tex}}
    \only<2>{\input{expose-F1.tex}}
    \only<3->{\input{expose-F1R2.tex}}
    \node[right] at (7, 2.5) {\large $\id
      \onslide<2->{* F}
      \onslide<3->{* R^2}$};
  \end{tikzpicture}
  \hfil
  \begin{minipage}[b]{0.48\linewidth}
    cube${}\leftrightarrow{}$séquence mouvements
    \par\vspace*{2ex}

    \noindent\hspace*{4em}
    \begin{tikzpicture}[x=0.8ex,y=0.8ex]
      \input{expose-id.tex}
      \node[right] at (13, 4.5) {$\id$};
    \onslide<2->{%
      \begin{scope}[yshift=-8ex]
        \input{expose-F1.tex}
        \node[right] at (13, 4.5) {$\id * F$};
      \end{scope}
    }
    \onslide<3->{%
      \begin{scope}[yshift=-16ex]
        \input{expose-F1R2.tex}
        \node[right] at (13, 4.5) {$\id * F * R²$};
      \end{scope}
    }
    \end{tikzpicture}
  \end{minipage}

  \vspace*{2ex}
  \onslide<4->%
  \begin{minipage}{0.9\linewidth}% do not hide the UMH logo
    \begin{block}{Notation}
      $g * X$ est le cube résultant du cube $g$ auquel on a appliqué le
      mouvement $X$.
    \end{block}
  \end{minipage}
\end{frame}


\section{L'ensemble des Rubik's cubes}
\begin{frame} \frametitle{L'ensemble de tous les Rubik's cubes}
 
  \begin{definition}
    $\G ={}$l'ensemble de toutes les configurations possibles du
    Rubik's cube i.e.\ toutes les configurations obtenues en
    appliquant une séquence quelconque de mouvements à $\id$.
  \end{definition}
 
  \begin{align*}
    \G = \Biggl\{
    &\cube{expose-id.tex} = \id, \quad
    \cube{expose-F1.tex} = \id * F, \quad
    \cube{expose-F1R2.tex} = \id * F * R^2, \\
    &\cube{expose-F1B2.tex} = \id * F * B^2 = \id * B^2 * F, \quad
    \dotsc\Biggr\}
  \end{align*}
 
  \onslide<2->%
  \vspace*{3ex}%
  Nombre d'éléments de $\G = 43.252.003.274.489.856.000
  \approx 43 \cdot 10^{18}$
\end{frame}

\begin{frame} \frametitle{Notion de mouvement inverse}
  \begin{tikzpicture}[x=2.6ex,y=2.6ex]
    \only<1>{\input{expose-id.tex}}
    \only<2>{\input{expose-R1.tex}}
    \only<3->{\input{expose-R1R3.tex}}
    \node[right] at (7, 2.5) {\large $\id
      \onslide<2->{* R}
      \onslide<3->{* R^3}$};
  \end{tikzpicture} 
  \hfil
  \begin{minipage}[b]{0.48\linewidth}
    \par\vspace*{2ex}  
    \noindent\hspace*{4em}
    \begin{tikzpicture}[x=0.8ex,y=0.8ex]
      \input{expose-id.tex}
      \node[right] at (13, 4.5) {$\id$};
      \onslide<2->{%
        \begin{scope}[yshift=-8ex]
          \input{expose-R1.tex}
          \node[right] at (13, 4.5) {$\id * R$};
        \end{scope}
      }
      \onslide<3->{%
        \begin{scope}[yshift=-16ex]
          \input{expose-R1R3.tex}
          \node[right] at (13, 4.5) {$\id * R * R^3$};
        \end{scope}
      }
    
    \end{tikzpicture}
  \end{minipage}

  \begin{minipage}{0.48\linewidth}
    \onslide<4->{%
      \par\vspace*{4ex} 
      \hspace*{3em} 
      On retrouve $\id$!
    }
  \end{minipage}
  \hfill
  \onslide<5->{%
  \begin{minipage}{0.48\linewidth}
    \begin{block}{Conclusion}
      $\id * R * R^3 = \id * R^4  = \id$ \\
      ou $R * R^3 = R^4 = \id$
    \end{block}
  \end{minipage}
}
\end{frame}

\begin{frame} \frametitle{Notion de mouvement inverse (suite)}
  \begin{tikzpicture}[x=2.6ex,y=2.6ex]
    \only<1>{\input{expose-scrambled.tex}}
    \only<2>{\input{expose-scrambledR1.tex}}
    \only<3->{\input{expose-scrambledR3.tex}}
    \node[right] at (7, 2.5) {\large $g
      \onslide<2->{* R}
      \onslide<3->{* R^3}$};
  \end{tikzpicture} 
  \hfil
  \begin{minipage}[b]{0.48\linewidth}
    \par\vspace*{2ex}  
    \noindent\hspace*{4em}
    \begin{tikzpicture}[x=0.8ex,y=0.8ex]
      \input{expose-scrambled.tex}
      \node[right] at (13, 4.5) {$g$};
      \onslide<2->{%
        \begin{scope}[yshift=-8ex]
          \input{expose-scrambledR1.tex}
          \node[right] at (13, 4.5) {$g * R$};
        \end{scope}
      }
      \onslide<3->{%
        \begin{scope}[yshift=-16ex]
          \input{expose-scrambledR3.tex}
          \node[right] at (13, 4.5) {$g * R * R^3$};
        \end{scope}
      }
    
    \end{tikzpicture}
  \end{minipage}

  \begin{minipage}{0.48\linewidth}
    \onslide<4->{%
      \par\vspace*{4ex} 
      \hspace*{3em} 
      On retrouve $g$!
    }
  \end{minipage}
  \hfill
  \onslide<5->{%
  \begin{minipage}{0.48\linewidth}
    \begin{block}{Conclusion}
      $g * R * R^3 = g * R^4  = g$ \\
      car $R * R^3 = R^4 = \id$
    \end{block}
  \end{minipage}
}
\end{frame}


\begin{frame} \frametitle{Notion de mouvement inverse (suite)}
\onslide<1->{%
  Qu'est-ce que l'inverse de $x \in \IR$?
 }
\onslide<2->{%
 \begin{block}{Définition}
   Soit $x \in \IR$.
   On dit que $y$ est l'\textit{inverse} de $x$ si $x \cdot y = 1$.
 \end{block}
}
\par\vspace*{2ex}
\onslide<3->{Analogie avec les mouvements:}
\onslide<4->{%
  \begin{block}{Définition}
    Soit $X$ un mouvement de base.
    Nous dirons que $Y$ est le mouvement \textit{inverse} de $X$
    si $X * Y = \id$.
  \end{block}
 }
\par\vspace*{2ex}
\onslide<5->{%
  On a vu que $R * R^3 = \id$. \\
  Donc $R^3$ est le mouvement inverse de $R$.
}
\end{frame}


\section{Le problème}
\begin{frame} \frametitle{Le problème}

  \begin{minipage}[t]{0.43\linewidth}
    \begin{block}{Première formulation}
      Étant donné un cube $g$, le remettre en \og position de base\fg.
    \end{block}
  \end{minipage}
  \hfill
  \onslide<5->{%
    \begin{minipage}[t]{0.47\linewidth}
      \begin{block}{Formulation mathématique}
        Étant donné  $g \in \G$, trouver des mouvements de base $X_1,
        \dotsc, X_n$ tels que $g * X_1 * \cdots * X_n = \id$.
      \end{block}
    \end{minipage}
  }

  \vspace*{3ex}%
  \textsc{Exemple:}
  \begin{center}
    \begin{tikzpicture}[x=1ex,y=1ex]
      \input{expose-F1R2U3.tex}
      \node[below] at (4.5, -0.5) {$g$};
      \onslide<2->%
      \begin{scope}[xshift=15ex]
        \draw[<-,thick] (-0.5, 4.5) -- +(-2.5,0) node[pos=0.5,above]{$U$};
        \input{expose-F1R2.tex}
        \node[below] at (4.5, -0.5) {$g * U$};
      \end{scope}
      \onslide<3->%
      \begin{scope}[xshift=30ex]
        \draw[<-,thick] (-0.5, 4.5) -- +(-2.5,0) node[pos=0.5,above]{$R^2$};
        \input{expose-F1.tex}
        \node[below] at (4.5, -0.5) {$g * U * R^2$};
      \end{scope}
      \onslide<4->%
      \begin{scope}[xshift=45ex]
        \draw[<-,thick] (-0.5, 4.5) -- +(-2.5,0) node[pos=0.5,above]{$F^3$};
        \input{expose-id.tex}
        \node[below] at (4.5, -0.5) {$g * U * R^2 * F^3 = \id$};
      \end{scope}
    \end{tikzpicture}
  \end{center}
\end{frame}

\section{Idée de l'algorithme}
\begin{frame} \frametitle{Idée de l'algorithme}
  \begin{block}{}
    Comment trouver la séquence de mouvements pour résoudre un cube~$g$ ?
  \end{block}

  Pas à pas :
  \par\vspace*{-2ex}
  \begin{center}
    \begin{tikzpicture}[x=1ex,y=1ex]
      \input{expose-F1R2U3.tex}
      \node[right] at (7.5, 2) {$g$};
      \draw[->] (4.5,-0.5) -- +(-18,-6) node[pos=0.5,above]{$F$};
      \begin{scope}[xshift=-20ex,yshift=-16ex]
        \input{expose-F1R2U3F1.tex}
        \node[right] at (7.5, 2) {$g * F$};
      \end{scope}
      \draw[->] (4.5,-0.5) -- +(1.5,-6) node[pos=0.5,left]{$U$};
      \begin{scope}[yshift=-16ex]
        \fill[black] (-2,4.5) circle(0.7pt) (-4,4.5) circle(0.7pt)
        (-6,4.5) circle(0.7pt);
        \input{expose-F1R2.tex}
        \node[right] at (7.5, 2) {$g * U$};
      \end{scope}
      \draw[->] (4.5,-0.5) -- +(21.5,-6) node[pos=0.5,above]{$D^3$};
      \begin{scope}[xshift=20ex,yshift=-16ex]
        \fill[black] (-2,4.5) circle(0.7pt) (-4,4.5) circle(0.7pt)
        (-6,4.5) circle(0.7pt);
        \input{expose-F1R2U3D3.tex}
        \node[right] at (7.5, 2) {$g * D^3$};
      \end{scope}
      \node[right] at (-20,-18.5) {%
        $\underbrace{\hspace*{52ex}}_{\text{18 configurations à examiner}}$
      };
    \end{tikzpicture}
  \end{center}

  \vspace*{-1ex}
  Trouvé l'identité ?
\end{frame}

\begin{frame} \frametitle{Idée de l'algorithme (suite)}

  \begin{block}{}
    Trouvé l'identité ?
    \begin{itemize}
    \item<2-> \textit{Oui} : on a fini
    \item<3-> \textit{Non} : on recommence !
    \end{itemize}
  \end{block}

  \onslide<3->%
  \begin{center}
    \footnotesize
    \begin{tikzpicture}[x=0.6ex,y=0.6ex]
      \input{expose-F1R2U3.tex}
      \node[right] at (7.5, 2) {$g$};
      \draw[->] (4.5,-0.5) -- +(-18,-6) node[pos=0.5,above]{$F$};
      \begin{scope}[xshift=-12ex,yshift=-10ex]
        \input{expose-F1R2U3F1.tex}
        \node[right] at (7.5, 1.5) {$g * F$};
        %% Level 3 for that node
        \draw[->] (4.5,-0.5) -- +(-7,-6);
        \draw[->] (4.5,-0.5) -- +(1.5,-6);
        \draw[->] (4.5,-0.5) -- +(7,-6);
        \node[below] at (4.5,-6.5) {\hbox to 10ex{\dotfill}};
      \end{scope}
      \draw[->] (4.5,-0.5) -- +(1.5,-6) node[pos=0.5,left]{$U$};
      \begin{scope}[yshift=-10ex]
        \fill[black] (-2,4.5) circle(0.7pt) (-4,4.5) circle(0.7pt)
        (-6,4.5) circle(0.7pt);
        \input{expose-F1R2.tex}
        \node[right] at (7.5, 1.5) {$g * U$};
        %% Level 3 for that node
        \draw[->] (4.5,-0.5) -- +(-7,-6);
        \draw[->] (4.5,-0.5) -- +(1.5,-6);
        \draw[->] (4.5,-0.5) -- +(7,-6);
        \node[below] at (4.5,-6.5) {\hbox to 10ex{\dotfill}};
      \end{scope}
      \draw[->] (4.5,-0.5) -- +(21.5,-6) node[pos=0.5,above]{$D^3$};
      \begin{scope}[xshift=12ex,yshift=-10ex]
        \fill[black] (-2,4.5) circle(0.7pt) (-4,4.5) circle(0.7pt)
        (-6,4.5) circle(0.7pt);
        \input{expose-F1R2U3D3.tex}
        \node[right] at (7.5, 1.5) {$g * D^3$};
        %% Level 3 for that node
        \draw[->] (4.5,-0.5) -- +(-7,-6);
        \draw[->] (4.5,-0.5) -- +(1.5,-6);
        \draw[->] (4.5,-0.5) -- +(7,-6);
        \node[below] at (4.5,-6.5) {\hbox to 10ex{\dotfill}};
      \end{scope}
      \node[right] at (-25,-28) {%
        $\underbrace{\hspace*{36ex}}_{18 \times 18 = 324 \text{ configurations à examiner}}$
      };
    \end{tikzpicture}
  \end{center}

  Trouvé l'identité ?
\end{frame}


\begin{frame} \frametitle{Idée de l'algorithme (suite)}

  \begin{minipage}{0.5\linewidth}
    \begin{math}
      \begin{array}{l|l<{{}}@{}r}
        \text{Niveau}& \multicolumn{2}{c}{\text{\# de configurations}}\\
        \hline
        \rule{0pt}{2.2ex}%
        1& 18^1 =& 18\\
        2& 18^2 =& 324\\
        3& 18^3 =& 5.832\\
        4& 18^4 =& 104.976\\
        5& 18^5 =& 1.889.568\\
        6& 18^6 =& 34.012.224\\
        7& 18^7 =& 612.220.032\\
        8& 18^8 =& 11.019.960.576
      \end{array}
    \end{math}

    \vspace*{2ex}%
    Explosion des cas !
  \end{minipage}
  \hfill
  \onslide<2->%
  \begin{minipage}{0.45\linewidth}
    \begin{block}{Ingrédients}
      \begin{itemize}
      \item<2-> Sous problèmes
      \item<3-> Multiplication rapide
      \item<4-> Recherche «~informée~» \& \emph{pruning}
      \item<5-> Réductions par symétries
      \end{itemize}
    \end{block}
  \end{minipage}
\end{frame}

\section{Références}
\begin{frame} \frametitle{Bibliographie}
  \begin{thebibliography}{}

  \bibitem{} The Perplexing Life of Erno Rubik, Discover magazine,
    mars~1986.
    \url{http://www.puzzlesolver.com/puzzle.php?id=29&page=15}


  \bibitem{} Herbert Kociemba, Cube Explorer pages,
    \url{http://kociemba.org/cube.htm}

  \bibitem{} Tomas Rokicki, Herbert Kociemba, Morley Davidson, John
    Dethridge,
    Every position of Rubik's Cube™ can be solved in twenty moves or less. 
    \url{http://www.cube20.org/}
  \end{thebibliography}
\end{frame}

\end{document}
%%% Local Variables:
%%% mode: latex
%%% TeX-master: t
%%% End:
